\documentclass{article}
\usepackage{amsmath}

\begin{document}

\section*{Exercício 6 - Seção 15.7}
\textbf{Enunciado:} Calcule o momento de massa em relação ao plano $xy$ do sólido $E$ que está dentro do cilindro $x^2 + y^2 = 4$ entre os planos $z = 0$ e $z = 1$, se a densidade em cada ponto é proporcional à distância ao eixo $z$.

\textbf{Resolução} Usando coordenadas cilíndricas, as integrais ficam:

\begin{align*}
M &= 2\int_0^{2\pi}\int_1^2\int_0^1 \rho^3 \sin^2\phi , d\rho , dz , d\theta
\end{align*}

Simplificando e resolvendo as integrais, temos:

\begin{align*}
M &= 2\int_0^{2\pi}\int_1^2\int_0^1 \rho^3 \sin^2\phi , d\rho , dz , d\theta \\
&= 2\int_0^{2\pi} , d\theta \int_1^2 , dz \int_0^1 \rho^3 \sin^2\phi , d\rho \\
&= 2\cdot 2\pi \cdot \frac{1}{3} \cdot \frac{1}{2} \\
&= \frac{2\pi}{3}
\end{align*}

Portanto, o momento de massa em relação ao plano $xy$ do sólido $E$ é $\frac{2\pi}{3}$.

\newpage

\section*{Exercício 8 - Seção 15.7}
\textbf{Enunciado:} Calcule a massa do sólido limitado pelo parabolóide $z = 1 - x^2 - y^2$ e pelo plano $z = 0$, onde a densidade em cada ponto é proporcional à distância ao plano $z = 0$.

\textbf{Resolução} Usando coordenadas cilíndricas, temos:

\begin{align*}
M &= \int_0^{2\pi}\int_0^1\int_0^{1-r^2} \rho \cdot \sqrt{1 + \frac{\partial z}{\partial \rho}^2 + \frac{\partial z}{\partial \phi}^2} , dz , d\rho , d\phi \\
&= \int_0^{2\pi}\int_0^1\int_0^{1-r^2} \rho \cdot \sqrt{1 + 4\rho^2} , dz , d\rho , d\phi \\
&= \frac{\sqrt{17}}{6} \cdot \pi
\end{align*}

Portanto, a massa do sólido é $\frac{\sqrt{17}}{6} \cdot \pi$.

\newpage

\section*{Exercício 30 - Seção 15.7}
\textbf{Enunciado:} Calcule o momento de inércia do sólido limitado pelo cone $z = \sqrt{x^2 + y^2}$ e pelo plano $z = 1$ em relação ao eixo $z$.

\textbf{Resolução} Usando coordenadas cilíndricas, temos:

\begin{align*}
I_z &= \iiint_E \rho^2 , dV \\
&= \int_0^{2\pi}\int_0^1\int_0^{r} \rho^3 \cdot \sin^2 \phi , dz , d\rho , d\phi \z\\
&= \frac{1}{4} \cdot \pi \int_0^1 r^4 , dr \z\\
&= \frac{1}{20} \cdot \pi
\end{align*}

Portanto, o momento de inércia do sólido em relação ao eixo $z$ é $\frac{1}{20} \cdot \pi$.
\end{document}
