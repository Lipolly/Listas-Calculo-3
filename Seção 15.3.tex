\documentclass{article}
\usepackage{amsmath}
\usepackage{amssymb}
\usepackage{geometry}

\begin{document}

\section*{Exercício 4 - Seção 15.3}

\textbf{Enunciado:} Calcular a integral dupla de $f(x,y) = y$ sobre a região $D$ limitada pela circunferência $x^2 + y^2 = 1$ e o eixo $x$ positivo, na primeira interseção com o eixo $x$.

\begin{align*}
    \iint_D f(x,y)\,dA &= \int_0^1 \int_0^{\pi} f(r\cos \theta, r\sin \theta) r\,d\theta\,dr \\
    &= \int_0^1 \int_0^{\pi} r\sin \theta \cdot r\,d\theta\,dr \\
    &= \int_0^1 \left[-r\cos \theta \right]_0^{\pi} r\,dr \\
    &= \int_0^1 2r^2\,dr \\
    &= \left[\frac{2}{3} r^3 \right]_0^1 \\
    &= \frac{2}{3}
\end{align*}

Portanto, a integral dupla de $f(x,y) = y$ sobre a região $D$ é $\frac{2}{3}$.

\newpage
\section*{Exercício 36 letra "a" - Seção 15.3}

\textbf{Enunciado:} Calcule a integral dupla $\iint_D e^{x^2 - y^2} dA$, onde $D$ é a região delimitada pelo triângulo com vértices $(0,0)$, $(1,0)$ e $(1,2)$.

\textbf{Solução.} A região $D$ é um triângulo, portanto podemos integrar sobre ele usando a integral dupla em coordenadas retangulares ou em coordenadas polares. Vamos usar coordenadas polares, pois a função integranda tem um termo da forma $x^2 - y^2$, que sugere a mudança de coordenadas. A equação da hipotenusa do triângulo é $y = 2x$, portanto a região $D$ pode ser descrita pelas seguintes desigualdades:

$$
\begin{aligned}
0 &\leq r \leq 2 \sin\theta \\
0 &\leq \theta \leq \frac{\pi}{4}.
\end{aligned}
$$

Assim, a integral dupla pode ser escrita como

$$
\iint_D e^{x^2 - y^2} dA = \int_0^{\frac{\pi}{4}} \int_0^{2\sin\theta} r e^{r^2 \cos 2\theta} dr d\theta.
$$

Fazendo a substituição $u = r^2$, $du = 2r dr$, obtemos

$$
\iint_D e^{x^2 - y^2} dA = \frac{1}{2} \int_0^{\frac{\pi}{4}} \int_0^{4\sin^2\theta} e^{u \cos 2\theta} du d\theta.
$$

Fazendo a substituição $v = u \cos 2\theta$, $dv = -2u \sin 2\theta d\theta$, obtemos

$$
\iint_D e^{x^2 - y^2} dA = -\frac{1}{4} \int_0^{\frac{\pi}{4}} e^{-\sin 2\theta} \left[ e^{-4 \sin^2 \theta} - 1 \right] d\theta.
$$

Podemos aproximar a integral acima usando um método numérico, ou podemos deixar a integral na forma acima como resposta. 

\end{document}
