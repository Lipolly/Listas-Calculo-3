\documentclass{article}
\usepackage{amsmath} % para símbolos matemáticos

\begin{document}

\section*{Exercício 2 - Seção 15.4}

\textbf{Enunciado:} Uma carga eletrica e distribuida sobre o disco $x^2 + y^2 \leq 1$, de modo que a densidade de carga em $(x,y)$ e $\rho(x,y) = \sqrt{x^2 + y^2}$ (medida em coulombs por metro quadrado). Determine a carga total no disco.

\begin{equation*}
Q = \iint_D \rho(x,y) , dA = \int_{0}^{2\pi} \int_{0}^{1} r\sqrt{r^2} , dr d\theta
\end{equation*}

\begin{equation*}
Q = \int_{0}^{2\pi} \int_{0}^{1} r^2 , dr d\theta = \frac{\pi}{2}
\end{equation*}

Portanto, a carga total no disco é $Q = \frac{\pi}{2}$ coulombs.

\end{document}
